\documentclass[12pt,a4paper]{article}
\usepackage[utf8]{inputenc}
\usepackage[T1]{fontenc}
\usepackage{lmodern}
\usepackage{listings}
\usepackage{xcolor}
\usepackage{booktabs}
\usepackage{hyperref}
\usepackage[margin=2.5cm]{geometry}

% Code styling
\definecolor{codegreen}{rgb}{0,0.6,0}
\definecolor{codegray}{rgb}{0.5,0.5,0.5}
\definecolor{codepurple}{rgb}{0.58,0,0.82}
\definecolor{backcolour}{rgb}{0.95,0.95,0.92}

\lstdefinestyle{mystyle}{
    backgroundcolor=\color{backcolour},   
    commentstyle=\color{codegreen},
    keywordstyle=\color{blue},
    numberstyle=\tiny\color{codegray},
    stringstyle=\color{codepurple},
    basicstyle=\ttfamily\footnotesize,
    breakatwhitespace=false,         
    breaklines=true,                 
    captionpos=b,                    
    keepspaces=true,                 
    showspaces=false,                
    showstringspaces=false,
    showtabs=false,                  
    tabsize=2
}
\lstset{style=mystyle}

\title{\textbf{NestJS Essential Guide}\\From Zero to CRUD}
\author{YouShop Project}
\date{\today}

\begin{document}

\maketitle
\tableofcontents
\newpage

\section{Installation \& Project Setup}

\subsection{Step 1: Install NestJS CLI}
\begin{lstlisting}[language=bash]
npm install -g @nestjs/cli
\end{lstlisting}

\subsection{Step 2: Create a New Project}
\begin{lstlisting}[language=bash]
nest new my-app
cd my-app
\end{lstlisting}

\subsection{Step 3: Install Validation Packages}
\begin{lstlisting}[language=bash]
npm install class-validator class-transformer
\end{lstlisting}

\subsection{Step 4: Configure Global Validation}
Edit \texttt{src/main.ts}:
\begin{lstlisting}[language=Java]
import { NestFactory } from '@nestjs/core';
import { ValidationPipe } from '@nestjs/common';
import { AppModule } from './app.module';

async function bootstrap() {
  const app = await NestFactory.create(AppModule);
  
  app.useGlobalPipes(new ValidationPipe({
    whitelist: true,           // Strip unknown properties
    forbidNonWhitelisted: true, // Error on unknown properties
    transform: true,           // Auto-convert types
  }));
  
  await app.listen(3000);
}
bootstrap();
\end{lstlisting}

\section{Core Concepts}

\subsection{The Module (The Organizer)}
Groups related code together. Every app has at least one: \texttt{AppModule}.

\begin{lstlisting}[language=Java]
// src/app.module.ts
@Module({
  imports: [ProductsModule],      // Other modules to use
  controllers: [AppController],   // Request handlers
  providers: [AppService],        // Business logic
})
export class AppModule {}
\end{lstlisting}

\subsection{The Controller (The Waiter)}
Handles HTTP requests and returns responses. Does NOT contain business logic.

\begin{lstlisting}[language=Java]
@Controller('products')          // Base route: /products
export class ProductsController {
  @Get()        // GET /products
  @Post()       // POST /products
  @Get(':id')   // GET /products/123
  @Patch(':id') // PATCH /products/123
  @Delete(':id')// DELETE /products/123
}
\end{lstlisting}

\subsection{The Service (The Chef)}
Contains all business logic. Called by Controllers.

\begin{lstlisting}[language=Java]
@Injectable()
export class ProductsService {
  create(data) { /* DB logic here */ }
  findAll() { /* DB logic here */ }
}
\end{lstlisting}

\subsection{The DTO (The Contract)}
Defines and validates the shape of incoming data.

\begin{lstlisting}[language=Java]
export class CreateProductDto {
  @IsString()
  name: string;

  @IsNumber()
  @Min(0)
  price: number;
}
\end{lstlisting}

\section{Build a Products CRUD}

\subsection{Step 1: Generate the Module}
\begin{lstlisting}[language=bash]
nest generate module products
nest generate controller products
nest generate service products
\end{lstlisting}

\subsection{Step 2: Create the DTO}
Create \texttt{src/products/dto/create-product.dto.ts}:
\begin{lstlisting}[language=Java]
import { IsString, IsNumber, Min, IsOptional } from 'class-validator';

export class CreateProductDto {
  @IsString()
  name: string;

  @IsString()
  @IsOptional()
  description?: string;

  @IsNumber()
  @Min(0)
  price: number;
}
\end{lstlisting}

Create \texttt{src/products/dto/update-product.dto.ts}:
\begin{lstlisting}[language=Java]
import { PartialType } from '@nestjs/mapped-types';
import { CreateProductDto } from './create-product.dto';

// Makes all fields optional
export class UpdateProductDto extends PartialType(CreateProductDto) {}
\end{lstlisting}

\textbf{Note:} Install \texttt{@nestjs/mapped-types} for \texttt{PartialType}:
\begin{lstlisting}[language=bash]
npm install @nestjs/mapped-types
\end{lstlisting}

\subsection{Step 3: Implement the Service}
Edit \texttt{src/products/products.service.ts}:
\begin{lstlisting}[language=Java]
import { Injectable, NotFoundException } from '@nestjs/common';
import { CreateProductDto } from './dto/create-product.dto';
import { UpdateProductDto } from './dto/update-product.dto';

@Injectable()
export class ProductsService {
  private products = []; // In-memory storage
  private idCounter = 1;

  create(dto: CreateProductDto) {
    const product = { id: this.idCounter++, ...dto };
    this.products.push(product);
    return product;
  }

  findAll() {
    return this.products;
  }

  findOne(id: number) {
    const product = this.products.find(p => p.id === id);
    if (!product) throw new NotFoundException(`Product #${id} not found`);
    return product;
  }

  update(id: number, dto: UpdateProductDto) {
    const product = this.findOne(id);
    Object.assign(product, dto);
    return product;
  }

  remove(id: number) {
    const index = this.products.findIndex(p => p.id === id);
    if (index === -1) throw new NotFoundException(`Product #${id} not found`);
    return this.products.splice(index, 1)[0];
  }
}
\end{lstlisting}

\subsection{Step 4: Implement the Controller}
Edit \texttt{src/products/products.controller.ts}:
\begin{lstlisting}[language=Java]
import { Controller, Get, Post, Body, Patch, Param, Delete, ParseIntPipe } from '@nestjs/common';
import { ProductsService } from './products.service';
import { CreateProductDto } from './dto/create-product.dto';
import { UpdateProductDto } from './dto/update-product.dto';

@Controller('products')
export class ProductsController {
  constructor(private readonly productsService: ProductsService) {}

  @Post()
  create(@Body() createProductDto: CreateProductDto) {
    return this.productsService.create(createProductDto);
  }

  @Get()
  findAll() {
    return this.productsService.findAll();
  }

  @Get(':id')
  findOne(@Param('id', ParseIntPipe) id: number) {
    return this.productsService.findOne(id);
  }

  @Patch(':id')
  update(@Param('id', ParseIntPipe) id: number, @Body() updateProductDto: UpdateProductDto) {
    return this.productsService.update(id, updateProductDto);
  }

  @Delete(':id')
  remove(@Param('id', ParseIntPipe) id: number) {
    return this.productsService.remove(id);
  }
}
\end{lstlisting}

\subsection{Step 5: Run \& Test}
\begin{lstlisting}[language=bash]
npm run start:dev
\end{lstlisting}

\textbf{Test with cURL:}
\begin{lstlisting}[language=bash]
# Create
curl -X POST http://localhost:3000/products \
  -H "Content-Type: application/json" \
  -d '{"name":"Laptop","price":999}'

# Read All
curl http://localhost:3000/products

# Read One
curl http://localhost:3000/products/1

# Update
curl -X PATCH http://localhost:3000/products/1 \
  -H "Content-Type: application/json" \
  -d '{"price":899}'

# Delete
curl -X DELETE http://localhost:3000/products/1
\end{lstlisting}

\section{Quick Reference: Common Decorators}

\begin{table}[h]
\centering
\begin{tabular}{@{}ll@{}}
\toprule
\textbf{Decorator} & \textbf{Purpose} \\
\midrule
\texttt{@IsString()} & Must be a string \\
\texttt{@IsNumber()} & Must be a number \\
\texttt{@IsEmail()} & Must be a valid email \\
\texttt{@IsNotEmpty()} & Cannot be empty \\
\texttt{@IsOptional()} & Field is optional \\
\texttt{@MinLength(n)} & Min string length \\
\texttt{@Min(n)} / \texttt{@Max(n)} & Min/Max number value \\
\texttt{@IsEnum(Enum)} & Must be from an enum \\
\texttt{@Type(() => Number)} & Transform to Number (class-transformer) \\
\bottomrule
\end{tabular}
\caption{Common class-validator Decorators}
\end{table}

\end{document}
